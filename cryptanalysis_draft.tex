\documentclass[12pt]{article}

\usepackage{amsfonts}
\usepackage{amsthm}
\usepackage{amsmath}
\usepackage{mathtools}
\usepackage{xcolor}
\usepackage{hyperref}
\usepackage{algorithm}% http://ctan.org/pkg/algorithm
\usepackage{algpseudocode}% http://ctan.org/pkg/algorithmicx

\newcommand{\floor}[1]{\lfloor #1 \rfloor}
\newcommand{\FF}{\mathbb{F}}
\newcommand{\ZZ}{\mathbb{Z}}
\newcommand{\LL}{\mathcal{L}}

\newtheorem{question}{Question}
\newtheorem{lemma}{Lemma}
\newtheorem{remark}{Remark}
\newtheorem{definition}{Definition}




\title{Cryptanalysis}
\author{ Sasha }



\begin{document}

\maketitle

\section{Information Set Decoding}

\cite{[DP19]}(just to not modify makefile added some references)

Notation: we take a vector $t \in \ZZ^{k}$, a matrix $A \in \ZZ^{k \times n}$. It can be reduced to its(maybe permuted) systematic form $H  = U * A = [I_{k}|D]$.


The goal is to find a vector $x \in \{0,1\}^{n}$ with small Hamming weight $h(x) = w$ that
\[
Ax = y
\]

Or equivalently $Hx = U*Ax = Uy =: t$, here we can use our knowledge about the shape of $H$.

What we do in this attack is an improvement of a bruteforce. The partition $x$ on two vectors $x_1 \in \{0,1\}^{k}$ and  $x_2  \in \{0,1\}^{n-k}$ so we have
\[
t = x_1 + D \cdot x_2
\]
We make a bet of the weight partition between $h(x_1) = w_1$ and $h(x_2) = w_2$, where $w_1 + w_2 = w$. Now we enumerate only the possible values of $x_2$, compute $x_1 = t - D \cdot x_2$ and check if it satisfies $h(x_1) = w_1$. If we don't find a correct pair with this weight distribution, we rerandomise H and t and start over.
\begin{algorithm}
\caption{ISD attack}\label{ISD}
\begin{algorithmic}[1]
\Procedure{Decoding}{$A,y$}
    \State Generate random matrix $U$
    \State $H \gets A * U$
    \State $t = y * U$
    \State Reduce A it's systematic form: $(I_k |D)$
    \For{$x_2$ of weight $w_2$}
        \State compute $x_1 = t - Dx_2$
        \If {$h(x_1) = w_1$ and contains only $0,1$}
        \State \Return $x = (x_1, x_2)$
        \EndIf
    \EndFor
\State \textbf{goto} \emph{line 2}.
\EndProcedure
\end{algorithmic}
\end{algorithm}
The average cost of such algorithm can be calculated as
\[
T = \frac{x_2\text{ bruteforce cost}}{Pr(w_2\text{ is a correct bet on the weight of }x_2)}
\]
Let us compute the values above.
The numerator:
\begin{itemize}
    \item Step 2: What is the cost of generating a random matrix?
    \item Step 3-4: $k*n^2 + k*n$
    \item Step 5: $k*(\log(q) + n-1) + k*(n-1)*(k-1)$
    \item Step 6-11: $(k*w_2 + k)*\binom{n-k}{w_2}$
\end{itemize}
In total we have: $k*(n^2 + 2n - 1 + \log(q) + (n-1)(k-1) + w_2\binom{n-k}{w_2} + \binom{n-k}{w_2})$
The denominator:
\[
\begin{split}
    Pr(w_2\text{ is a correct bet on the weight of }x_2) & = Pr(h(x_2) = w_2|h(x) = w) \\
    & = \frac{\binom{n-k}{w_2} \cdot \binom{k}{w_1}}{\binom{n}{w}}
\end{split}
\]
Therefore
\[
T = \frac{k*\binom{n}{w}*(n^2 + 2n - 1 + \log(q) + (n-1)(k-1) + w_2\binom{n}{w_2} + \binom{n}{w_2})}{\binom{k}{w_1}\binom{n-k}{w_2}}
\]
\subsection{Ternary case}
The numerator:
Only the number of choices for $x_2$ change
\[
    k*(n^2 + 2n - 1 + \log(q) + (n-1)(k-1) + w_2\binom{n-k}{w_2}*2^{w_2} + \binom{n-k}{w_2}*2^{w_2})
\]
The denominator:
\[
\begin{split}
    Pr(w_2\text{ is a correct bet on the weight of }x_2) & = Pr(h(x_2) = w_2|h(x) = w) \\
    & = \frac{\binom{n-k}{w_2} \cdot \binom{k}{w_1}}{\binom{n}{w}}
\end{split}
\]
Therefore
\[
T = \frac{k*\binom{n}{w}*(n^2 + 2n - 1 + \log(q) + (n-1)(k-1) + w_2\binom{n}{w_2}*2^{w_2} + \binom{n}{w_2}*2^{w_2})}{\binom{k}{w_1}\binom{n-k}{w_2}}
\]
\begin{lemma}
To minimize the average cost we take $w_1 = ???$ calculated with a script
\end{lemma}
\section{Meet in the Middle}
In this attack we have the same goal but no information about the form of the matrix A.
We partition $x$ and $A$ on two equal parts: $A = [A_1 | A_2]$, $x = x_1 | x_2$.
Then $Ax = y$ is equivalent to
\[
    A_1x_1 + A_2x_2 = y
\]
If we can find vectors $x_1$ and $x_2$ for which values $A_1x_1$ and $y - A_2x_2$ coincide
and sum of their weights is equal to $w$ they form a solution to our problem.


\begin{algorithm}
\caption{MitM attack}\label{MitM}
\begin{algorithmic}[1]
\Procedure{Decoding}{$A,y$}
    \State Generate random matrix $U$
    \State $(A_1|A_2) \gets A * U$
    \State $t = y * U$
    \For{$x_1$ of weight $\frac{w}{2}$}
    \State compute $A_1x_1$
    \State store in the hash table
    \EndFor
    \For{$x_2$ of weight $\frac{w}{2}$}
    \State compute $t - A_2x_2$
    \State look up in the hash table for a collision
    \If {collision found}
        \State \Return $x = (x_1, x_2)$
    \EndIf
    \EndFor
\State \textbf{goto} \emph{line 2}.
\EndProcedure
\end{algorithmic}
\end{algorithm}


Here we bet the weight is distributed equaly on both sides. So the average cost can be calculated as follows:
\[
  T = \frac{\text{cost of finding a collision}}{Pr(h(x_1) = h(x_2) = w/2| h(x) =w)}
\]
Numerator:
\begin{itemize}
    \item Step 2: What is the cost of generating a random matrix?
    \item Step 3-4: $k*n^2 + k*n$
    \item Step 5-8: $(k*\frac{w}{2}*\binom{n/2}{w/2}$
    \item Step 9-15: $(k*\frac{w}{2} + k)*\binom{n/2}{w/2}$
\end{itemize}

In total:

TIME COST = $k*(n^2 + n + (1 + w)\binom{n/2}{w/2})$

We need to store $(A_1x_1, x_1)$ for every $x_1$

MEMORY COST = $\binom{n/2}{w/2} * \text{(how do we count the size of it)}$

Denominator:
$Pr(h(x_1) = h(x_2) = w/2) = \frac{\binom{n/2}{w/2}^{2}}{\binom{n}{w}}$

Total cost(only time):
\[
  T = \frac{k*\binom{n}{w}*(n^2 + n + (1 + w)\binom{n/2}{w/2})}{\binom{n/2}{w/2}^2}
\]

\begin{question}
  how do we rerandomise in this case? $\rightarrow$ We can just multiply by any matrix!
\end{question}

\subsection{Ternary case}
Numerator:

TIME COST = $k*(n^2 + n + (1 + w)\binom{n/2}{w/2}*2^{w/2})$

We need to store $(A_1x_1, x_1)$ for every $x_1$

MEMORY COST = $\binom{n/2}{w/2} * 2^{w/2} * \text{(how do we count the size of it)}$

Denominator:
$Pr(h(x_1) = h(x_2) = w/2) = \frac{\binom{n/2}{w/2}^{2}}{\binom{n}{w}}$

Total cost(only time):
\[
  T = \frac{k*\binom{n}{w}*2^{w/2}*(n^2 + n + (1 + w)\binom{n/2}{w/2})}{\binom{n/2}{w/2}^2}
\]
\section{ISD + MitM}
Let us return to the case when $A$ is reduced to the systematic form $H = U * A = [I_{k}| D_1 | D_2]$ we partition $x = (x_0 |x_1 |x_2)$ on three vectors $x_0 = \in \{0,1\}^{k}$, $x_1,x_2 \in \{0,1\}^{\frac{n-k}{2}}$. Then
\[
  Ax = x_0 + D_1x_1 + D_2x_2
\]

We make bet that $h(x_0) = w_1$, $h(x_1) = h(x_2) = \frac{w_2}{2}$ and perform Meet-in-the-Middle attack trying to find approximate collisions between $D_1x_1$ and all possible $t - D_2x_2$

For that we desing a compression function $f$ that will often map close vectors to the same value. It operates as follows:
\[
\forall v = (v_1, \dots , v_k)\in \ZZ_{q}^{k}: f_{p}(v) = (v_1 // p, v_2 // p, \dots, v_k // p)
\]
Here by $//$ we mean a quotient after euclidean division by $p$, $p$ is a parameter of the function.

We store a table of $f(D_1x_1)$ in memory and look-up there for $t - D_2x_2$. We will need to deal with false positives and false negatives in this collision search. Too many false positives can increase the cost of the search for every randomization of the public key matrix and too many false negatives can increase the number of randomizations itself. Let us estimate the cost of running our attack.
\begin{algorithm}
\caption{ISD+MitM attack}\label{algo}
\begin{algorithmic}[1]
\Procedure{Decoding}{$A,y$}
    \State Generate random matrix $U$
    \State $A \gets A * U$
    \State $t = y * U$
    \State Reduce A it's systematic form: $(I_k |D_1|D_2)$
    \For{$x_1$ of weight $\frac{w_2}{2}$}
    \State compute $f_{p}(D_1x_1)$
    \State store in the hash table
    \EndFor
    \For{$x_2$ of weight $\frac{w_2}{2}$}
    \State compute $f_{p}(t - D_2x_2)$
    \State look up in the hash table for a collision
    \If {collision found}
        \State compute $x_0 = t - D_2x_2 - D_1x_1$
        \If {$h(x_0) = w_1$ and contains only $0,1$}
        \State \Return $x = (x_0, x_1, x_2)$
        \EndIf
    \EndIf
    \EndFor
\State \textbf{goto} \emph{line 2}.
\EndProcedure
\end{algorithmic}
\end{algorithm}
To ease the analysis we present pseudocode of the algorithm. \ref{algo}

Computational cost:
Let us first compute the cost of running the algorithm when we were lucky to generate such matrix $U$ that we end up finding vector $x$ that satisfies all the coditions.
\begin{itemize}
    \item Step 2: What is the cost of generating a random matrix?
    \item Step 3-4: $k*n^2 + k*n$
    \item Step 5: $k*(\log(q) + n-1) + k*(n-1)*(k-1)$
    \item Step 6-9: $(k*\frac{w_2}{2} + k)*\binom{(n-k)/2}{w_2/2}$
    \item Step 10-19: $(k*\frac{w_2}{2} + k + k)*\binom{(n-k)/2}{w_2/2} + k*(\{\text{\# false positives}\} + 1)$
\end{itemize}
In total we have: $k*(n^2 + 2n + \log(q) + (n-1)(k-1) + w_2\binom{(n-k)/2}{w_2/2} + 3\binom{(n-k)/2}{w_2/2} + E(\{\text{\# false positives}\}))$

Memory cost:
We only store in memory the hash table of $(D_1x_1, x_1)$ in a cell with index $f_{p}(D_1x_1)$ for every $x_1$.
What is the size of every coordinate?

Total average cost:
\[
    T = \frac{k*(n^2 + 2n + \log(q) + (n-1)(k-1) + w_2\binom{(n-k)/2}{w_2/2} + 3\binom{(n-k)/2}{w_2/2} + E(\{\text{\# false positives}\}))}{Pr(\text{weight distribution bet was successful, there was no false negative in the collision search})}
\]
Let us calculate missing parts of the formula:
\[
\begin{split}
    E(\{\text{\# false positives}\}) = \binom{n}{w_2}*Pr(\text{false positive})
\end{split}
\]
\[
\begin{split}
    Pr(\text{false positive}) & = (1 - Pr(\text{collision corresponds to a correct looking } x_0| \text{  collision happened})) \\
    & * Pr(\text{collision happens}) \\
    & = \Big(1 - \binom{n}{w_1}\frac{p-1}{p^2}^{w_1}\frac{1}{p}^{n-w_1}\Big)\frac{1}{q//p}^{2n}
\end{split}
\]
\[
\begin{split}
    Pr(\text{false negative}) & = Pr(\text{for some coordinate the +1 difference is located on the edge of the box} \\
    & \text{and for every other one it's still +1}) \\
    & = \frac{1}{q}^{n} - \frac{q-b}{q^2}^{n}
\end{split}
\]
\[
\begin{split}
    Pr(\text{weight distribution is correct}) & = Pr(h(x_0) = w_1, h(x_1) = h(x_2) = \frac{w_2}{2}) \\
    & = \frac{\binom{n-k/2}{w_2/2}^{2}\binom{k}{w_1}}{\binom{n}{w}}
\end{split}
\]
Putting everything together we obtain the total average cost.

\subsection{Ternary case}

\section{Question 5 and 6}


\bibliography{cryptanalysis_draft}
\bibliographystyle{ieeetr}

\end{document}
