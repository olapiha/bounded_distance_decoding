\documentclass[12pt]{article}

\usepackage{amsfonts}
\usepackage{amsthm}
\usepackage{amsmath}
\usepackage{amssymb}
\usepackage{mathtools}
\usepackage{xcolor}
\usepackage[normalem]{ulem}
\usepackage{hyperref}
\hypersetup{
    colorlinks=true,
    linkcolor=blue
}


\newcommand{\floor}[1]{\lfloor #1 \rfloor}
\newcommand{\ZZ}{\mathbb{Z}}
\newcommand{\LL}{\mathcal{L}}
\DeclareMathOperator{\Span}{Span}


\newtheorem{question}{Question}
\newtheorem{lemma}{Lemma}

\newtheorem{theorem}{Theorem}
\newtheorem{remark}{Remark}
\newtheorem{definition}{Definition}


\title{Comparing Families of Lattices for efficient Bounded Distance Decoding near Minkowski’s Bound}

\author{ Oleksandra Lapiha }
\usepackage{xcolor}
\def\added{\bgroup \markoverwith{\textcolor{green!50!blue}{\lower4.5pt\hbox{\sixly \char58}}}\ULon}
\def\toimprove{\bgroup \markoverwith{\textcolor{red}{\lower4.5pt\hbox{\sixly \char58}}}\ULon}
\def\improved{\bgroup \markoverwith{\textcolor{black}{\lower4.5pt\hbox{\sixly \char58}}}\ULon}


\begin{document}

\maketitle

\section{Introduction}
Plan:
\begin{enumerate}
    \item BDD problem +
    \item Short description of the initial construction(a feel for how our lattices are going to look like)(Maybe give the definition of the lattice as well?)
    \item Description of the decoding algorithm
    \item The purpose of this work
\end{enumerate}
Explain somewhere why $l_1$ metric is natural way of thinking about the problem.

\textbf{Decoding problem.} Message decoding is a problem that arises when two parties need to communicate over a noisy channel. Here we are in the setting where we only care about integrity of our data and we assume there's no adversary listening and modifying messages in the network. We are fighting information loss that occurs due to our infrastructure being slightly disturbed by forces of nature.

To achieve that we encode messages as points in Euclidean space. Now we can model "disruption" as distance between input and output point. It is "slight" if can be bounded. A natural way to arrange all messages is a lattice(all integer linear combinations of a linearly independent set of vectors). The decoding procedure aims to identify the point of the lattice closest to the output point. We assume that closest lattice point is indeed the input message. Minimal distance between points is bounded by the length of the shortest vector $\lambda_{1}(\LL)$ of the lattice. If the error exceeds its half closest point is not the input message anymore so decoding becomes impossible.

We want an algorithm where the decoding radius is close to $\frac{\lambda_{1}(\LL)}{2}$. Unfortunately, Lattice Shortest Vector Problem(SVP) is believed to be hard in general case, so we cannot evaluate exactly how good our decoding radius is. Nevertheless, we have a bound on the shortest vector itself. So we aim to get a decoding radius close to it instead.
\begin{theorem}[Minkowski's First Theorem]
    For any full-rank lattice $\LL$ of rank $n$,
    \[
        \lambda_{1}(\LL) \leq \sqrt{n} \det(\LL)^{1/n}
    \]
\end{theorem}
\textbf{Some name of the subsection.}





In this work by efficient algorithm we mean an algorithm that runs in polynomial time.


\section{Generalization of the construction for integers}
In this section we present first generalization of the decoding algorithm introduced in \cite{[DP19]}. In their paper Léo Ducas and Cécile Pierrot use properties of the group $(\ZZ/m\ZZ)^*$ for modulus $m$ which is a prime power. The decoding radius achievable in polynomial-time depends on the ratio
\[
\frac{m}{\varphi(m)}
\]
The higher the ratio the larger the achievable decoding radius is. In our generalization we take $m'$ an arbitrary product of prime powers. If $m$ and $m'$ are of the same size, as $m'$ is smoother $\varphi(m')$ will be smaller which gives us a better decoding radius.

In the construction for $m$ prime power authors compute lattice basis directly using the fact that $(\ZZ/m\ZZ)^*$ is cyclic and we can calculate discrete logarithms of its elements.

Our result is a deterministic efficient algorithm for computation of the basis for any integer $m$. We find a way to deal with the structure of the group $(\ZZ/m\ZZ)^*$ in a different way and compute lattice basis through its dual.


\subsection{Definition of discrete logarithm lattice }
\label{definition section}
In this chapter we take $m = \prod_{i=1}^{k} q_{j}^{e_{j}}$ where $\{q_{j}\}$ are odd prime numbers and $\{e_{j}\}$ are positive integers. We work with a group $(\ZZ/m\ZZ)^*$ which is not cyclic anymore so we cannot exploit properties of discrete logarithm right away. Nevertheless, the Chinese Remainder Theorem (CRT) gives us the structure of the group:
\[
    (\ZZ/m\ZZ)^* \sim \prod_{j=1}^{k}(\ZZ/q_{j}^{e_{j}}\ZZ)^*
\]
For every prime $q_{j} > 2$ and every $e_{j} \geq 1$ the group $(\ZZ/q_{j}^{e_{j}}\ZZ)^*$ is known to be cyclic. So now, we can consider discrete logarithms in every component of the product to \toimprove{find a lattice basis}.\footnote{Not clear yet what is the lattice you are talking about. Maybe you can swap this paragraph with the definition that follows. When done, you can then say "the lattice $\LL$" rather than just "the lattice".}

Similarly to the initial construction we take numbers $n$ and $B$ such that we can find $n$ primes $p_{1}, \dots , p_{n}$ different from every $q_{j}$ and bounded by $B$. We consider a group morphism:
\[
    \psi : \ZZ^{n} \rightarrow (\ZZ/m\ZZ)^*
\]
\[
    (x_{1}, \dots, x_{n}) \mapsto \prod_{i=1}^{n}p_{i}^{x_{i}} \pmod{m}\footnote{Since  $p_{1}, \dots , p_{n}$ are relatively prime with $m$ for every element of $\ZZ^{n}$ its image is indeed a part of $ (\ZZ/m\ZZ)^*$.}
\]


The kernel of $\psi$ is a subgroup of $\ZZ^{n}$ so it is a lattice. We will call it discrete logarithm lattice from now on.
\[
    \LL \coloneqq \ker \psi = \{(x_{1}, \dots, x_{n}) \in \ZZ^{n} | \prod_{i=1}^{n}p_{i}^{x_{i}} \equiv 1 \pmod{m}\}
\]
Appying the CRT gives us the following equivalence
\[
    \LL = \ker \psi = \{(x_{1}, \dots, x_{n}) \in \ZZ^{n} |  \forall 1 \leq j \leq k: \prod_{i=1}^{n}p_{i}^{x_{i}} \equiv 1 \pmod{q_{j}^{e_{j}}}\}
\]
Or going even further, suppose we know for every $j$ a generator ${\beta_{j}}$ of  $(\ZZ/q_{j}^{e_{j}}\ZZ)^*$. Using the morphism between cyclic group and group of exponents we get a representation in terms of conditions on the exponents:
\[
    \LL = \{(x_{1}, \dots, x_{n}) \in \ZZ^{n} |  \forall 1 \leq j \leq k: \sum_{i=1}^{n}x_{i}\log_{\beta_{j}}p_{i}\equiv 0 \pmod{\varphi(q_{j}^{e_{j}})}\}
\]
Note that discrete logarithm functions that we are using have different input and output domains
\[
    \forall 1 \leq j \leq k: \log_{\beta_{j}}: (\ZZ/q_{j}^{e_{j}}\ZZ)^* \rightarrow (\ZZ/\varphi(q_{j}^{e_{j}})\ZZ)
\]
This is almost a parity check representation of $\LL$ except we have many parity check type conditions at the same time. Therefore, $\LL$ is an intersection of lattices $\LL_{j}$ where each of them is defined as
\[
\label{parity check}
    \LL_{j} \coloneqq \{(x_{1}, \dots, x_{n}) \in \ZZ^{n} | \sum_{i=1}^{n}x_{i}\log_{\beta_{j}}p_{i}\equiv 0 \pmod{\varphi(q_{j}^{e_{j}})}\}
\]

\subsection{Computing lattice basis}
Our goal is to compute a basis of discrete log lattice to be able to encode messages as its points afterwards. The idea of our algorithm is to compute a basis of the dual lattice first. And then obtain primal one from the dual.
Let us remind a definiton of a dual lattice and a dual basis.
\begin{definition}
    For a lattice $\LL \subseteq \mathbb{R}^{n}$ we define $\LL^{*} \subseteq (\mathbb{R}^{n})^{*}$ as a lattice of all linear maps $f:\mathbb{R}^{n} \rightarrow \mathbb{R}$ such that every lattice point is mapped to an integer value.
\end{definition}
Linear maps can be represented as inner poduct with a fixed vector. So equivalently
\[
    \LL^{*} = \{y \in \mathbb{R}^{n} | \forall x \in \LL:  \langle y,x\rangle \in \ZZ \}
\]

\begin{definition}
    For a basis $B = (b_{1}, \dots, b_{n}) \in \mathbb{R}^{m \times n}$, define the dual basis $D = (d_{1}, \dots, d_{n}) \in \mathbb{R}^{m \times n}$ as the unique basis that satisfies
    \begin{itemize}
        \item $\Span(D) = \Span(B)$
        \item $B^{T}D = I$
    \end{itemize}
\end{definition}
If B is an invertible square matrix $(B^{T})^{-1}$ satisfies the definition. In the case a non square matrix one can easily verify that $D = B(B^{T}B)^{-1}$ is the dual basis. \footnote{Since columns of $B$ are basis vectors, $B$ is a full column rank matrix. Therefore, $B^{T}B$ is always intertible.}


 Our algorithm \label{algorithm} has the following steps
\begin{enumerate}
    \item \label{step1} Calculate parity check representations for every $\LL_{j}$
    \item \label{step2} Get dual generating set of their intersection
    \item \label{step3} Eliminate linear dependencies in the generating set
    \item \label{step4} Obtain the basis of the primal lattice from the dual
\end{enumerate}
We will describe each of them in more details

\subsubsection{Parity check representations (\ref{step1})}
In the section \ref{definition section} we were able to represent $\LL$ as an intesection of $\LL_{1}, \dots, \LL_{k}$ for which we have parity check representations:
\[
    \LL_{j} \coloneqq \{(x_{1}, \dots, x_{n}) \in \ZZ^{n} | \sum_{i=1}^{n}x_{i}\log_{\beta_{j}}p_{i}\equiv 0 \pmod{\varphi(q_{j}^{e_{j}})}\}
\]
To compute them we calculate many discrete logarithms in finite groups. As you probably know it is a hard problem for a general case of which no polynomial algorithm is known. We need to choose parameters such that we can compute them efficiently. We discuss it in the section (\ref{complexity section}) of this document.

\subsubsection{Dual generating set (\ref{step2})}
To justify this step we need two lemmas
\begin{definition}
    For two lattices $\LL_1$, $\LL_2$ we define their sum
\[
    \LL_1 + \LL_2 \coloneqq \{x + y | x \in \LL_1, y \in \LL_2\}
\]
\end{definition}
This space $\LL_{1} + \LL_{2}$ can be generated by concatenation of bases of $\LL_{1}$ and $\LL_{2}$. It is a sum of two additive subgroups of $\mathbb{R}^{n}$ so it stays an additive subgroup. But it is not always discrete so it doesn't nessesarily form a lattice.

\begin{lemma}
    \footnote{Check it again with fresh head}
    \label{intersection lemma}
    Supose $\LL = \bigcap_{j=1}^{k} \LL_{j} \neq \{0\}$ and $\LL^{*}$, $\LL_{j}^{*}$ are duals of the respective lattices. Then $\LL^{*} = \sum_{j=1}^{k} \LL_{j}^{*}$
\end{lemma}
\begin{proof}
    Let us start from the end. Take lattices $\LL_{1}, \dots, \LL_{n}$ what is the dual of their sum.
    \footnote{We can take all $x_{i}$ equal to 0 but 1 of them}
\[
\begin{split}
(\sum_{j=1}^{n}\LL_j)^{*} & = \{y \in \ZZ^{n} | \forall x_{1} \in \LL_1, \dots, \forall x_{n} \in \LL_n: \langle y, \sum_{j=1}^{n} x_{j} \rangle \in \ZZ \} \\
& = \{y \in \ZZ^{n} | \forall 1 \leq j \leq n, \forall x_{j} \in \LL_j: \langle y,  x_{j} \rangle \in \ZZ \}
\end{split}
\]

So $y$ must be an element of every $\LL_{j}^{*}$. Therefore, $(\sum_{j=1}^{n}\LL_j)^{*} = \bigcap_{j=1}^{k} \LL_{j}^{*}$. Applying this assertion to lattices $\LL_{1}^{*}, \dots, \LL_{n}^{*}$ we have
\[
(\sum_{j=1}^{n}\LL_j^{*})^{*} = \bigcap_{j=1}^{k} \LL_{j} = \LL
\]
Now taking dual lattices of both sides of the equation we obtain:
\[
(\sum_{j=1}^{n}\LL_j^{*})^{**} = \LL^{*}
\]

\end{proof}

\begin{lemma}
    Let B be a square matrix, $B \in \mathbb{R}^{n \times n}$. Suppose we are given parity check representation of a lattice $\LL = \{x \in \ZZ^{n} | Bx \equiv 0 \pmod{p}\}$
    Then \textbf{rows} of the matrix
    \[
    \binom{\frac{1}{p} \cdot B}{I_{n}}
    \]
    form a generating set of the dual lattice.
\end{lemma}
\begin{proof}
    Another equivalent definition for $\LL$ would be:
    \[
        \LL  = \{x \in \ZZ^{n} | \frac{1}{p}Bx \equiv 0 \pmod{1}\}
    \]
Therefore, can represent $\LL$ as an intersection of the following lattices:
\[
    \LL_{1}  = \ZZ^{n}
\]
\[
    \LL_{2}  = \{x \in \mathbb{R}^{n} | \frac{1}{p}Bx \in \ZZ^{n} \}
\]
Then from lemma \ref{intersection lemma}
\[
    \LL^{*}  = (\LL_{1} \cap \LL_{2})^{*} = \LL_{1}^{*} + \LL_{2}^{*}
\]
It is obvious that $(\ZZ^{n})^{*} = \ZZ^{n}$. To prove that $(\frac{1}{p}B)^{T}$(here basis vectors are columns) is a basis of the dual it is enought to show $(\frac{1}{p}B)^{-1}$ is basis of the primal lattice. This is quite simple:
\[
    \forall x \in \ZZ^{n}: \frac{1}{p}B \cdot (\frac{1}{p}B)^{-1} \cdot x \in \ZZ
\]
And the other way:
\[
    \forall x \in \LL: \frac{1}{p}B \cdot x = y \in \ZZ^{n} \implies x = (\frac{1}{p}B)^{-1} \cdot y , y \in \ZZ^{n}
\]
A generating set of the sum of lattices can be obtained by concatenation of their bases, so we obtain our desired result.
\end{proof}

The algorithm takes parity check representations of every lattice $\LL_{j}$ scales them and adds an identity matrix. Output is the concatenation of calculated generating sets.

\subsubsection{Eliminating linear dependencies using elementary martix transformations (\ref{step3})}

Now we have obtained a generating set of the dual lattice but we would like to have its basis. The idea is to transform the matrix to its row echelon form so the resulting set has some zero vectors which we will discard and the basis of the lattice.

Out of all elementary matrix transformations we are only allowed adding to a row an integer mulpitle of another row, interchanging two rows, multiplying a row by -1. These three possibilities are called unimodular transformations. As you may have noticed the only restriction is that we can't mutiply a row by an integer different from $\pm 1$. This would result in a sublattice of $\LL$ with a higher determinant.

We use an algorithm for reducing the matrix to its Hermite normal form. It is an equivalent of row echelon form for matrices over $\ZZ$ with a restriction to unimodular transformations. Our input matrix can have rational coefficients so we first transform them into integers multiplying by the least common multiple of all denominators. We divide by it when matrix is in the Hermite form.

Resulting vectors are going to be quite long. If we want to control the size of the basis we can use LLL algorithm instead. This change will for example make our encoded messages shorter and easier to decode.

\subsubsection{Primal basis from dual basis (\ref{step4})}

A way to calculate a basis of primal lattice having the basis of the dual is straight forward having its definition.


\subsection{Complexity analysis}
\label{complexity section}
The first step (\ref{step1}) of basis computation algorithm (\ref{algorithm}) boils down to many computations of discrete logarithm in a finite group. For every distinct prime factor $q_{j}^{e_{j}}$ of the modulus we compute $n$ discrete logarithms(one for every prime $p_{i}$). It is $n \cdot k$ operations in total. Let us refer to $(\ZZ/q_{j}^{e_{j}}\ZZ)^*$ as group $G$. Group order is equal to $|G| = \varphi(q_{j}^{e_{j}}) =  q_{j}^{e_{j}-1}(q_{j}-1) = \prod_{i=1}^{k} t_{i}^{a_{i}}$.
It is a $q$-smooth integer, so to efficiently calculate discrete logarithms in this group we can use a combinaion of of Pohlig-Hellman \cite{[PH78]} and Pollard-$\rho$ \cite{[Pol78]} algorithms. Overall complexity in group operations is
\[
    O(\sum_{i=1}^{k} a_{i}(\ln(|G|) + \sqrt{t_{i}}))
\]
To be polynomial in lattice dimention $a_{i} \leq e_{j}-1$ and $t_{i} \leq q_{j}$ need to be polynomial in $n$. For this reason we configure $e_{j} = n$ for every $j$. As we have $n \cdot k$ such operations, $k$ should be polynomial in $n$.

%\begin{question}
%    Should I talk about how we find generators?(I don't really know how the default function in sagemath does it and what is the complexity, but I could discuss how it happens in theory)
%\end{question}
%{\bf Why not. }

It is easy to see that step two (\ref{step2}) takes linear time in $n + k$. Step three (\ref{step3}) is LLL algorithm, which is proven to run in polynomial time in $n + k$.

Finally, step four (\ref{step4}) includes matrix multiplication and inversion for matrices which dimention is bounded by $n + k$. They obviously take polynomial time.

We still haven't configured parameters $k, q_{j}, p_{i}$. $q_{j}, p_{i}$ must be pairwise different and as small as possible, so we need $n+k$ distinct prime numbers. We choose $q_{j}$ to be equal to first $k$ odd prime numbers and $p_{i}$ equal to $2$ and next $n-1$ smallest primes.
\subsection{Was it all worth it?}
Decoding algorithm extends naturally to the generalized setting. Let us remind that nomalized decoding $l_{1}$-radius was equal to
\[
    \bar r^{(1)} = \frac{\ln(m/2)}{4 \varphi(m)^{1/n} \cdot \ln(B)}
\]
where $B$ is a constant bounding all $p_{i}$
As it turns out factorization of $m$ does not affect the decoding radius substantially. It can only introduce a minor improvement. We prove that with a following lemma:

\begin{lemma}
Assume the primes $p_{1}, \dots, p_{n}$ are the same for both lattices. If $a \cdot m \sim b \cdot m' \sim e^n$, when $n \rightarrow \infty$ Then
\[
    \frac{\bar r_{m}^{(1)}}{\bar r_{m'}^{(1)}} \rightarrow 1, n \rightarrow \infty
\]
\end{lemma}
\begin{proof}
Since the primes are the same, they are also bounded by the same constant $B$ so the term $\frac{1}{4 \ln(B)}$ cancels out in the numerator and denominator.
\[
\begin{split}
\frac{\bar r_{m}^{(1)}}{\bar r_{m'}^{(1)}}
& = \frac{\ln(m/2) \cdot \varphi(m')^{1/n}}{\ln(m'/2) \cdot \varphi(m)^{1/n}} \\
& = \frac{\ln(e^n/2a) \cdot \varphi(m')^{1/n}}{\ln(e^n/2b) \cdot \varphi(m)^{1/n}} \\
& = \frac{\varphi(m')^{1/n}}{\varphi(m)^{1/n}}
\end{split}
\]
We apply a few classic results to estimate the growth of $\varphi(n)$ \cite{[HW09]}:
\[
    \limsup\limits_{n \rightarrow \infty} \frac{\varphi(n)}{n} = 1
\]
\[
    \liminf\limits_{n \rightarrow \infty} \frac{\varphi(n) \cdot \ln\ln(n)}{n} = e^{-\gamma}
\]
where $\gamma$ is the Euler constant, $e^{-\gamma} = 0.56145948\dots$
\[
\begin{split}
\lim_{n \rightarrow \infty} \frac{\varphi(m')^{1/n}}{\varphi(m)^{1/n}}
& \leq \frac{\limsup\limits_{n \rightarrow \infty} \varphi(m')^{1/n}}{\liminf\limits_{n \rightarrow\infty}\varphi(m)^{1/n}} \\
& = \lim_{n \rightarrow \infty} \bigg(\frac{m \cdot \ln\ln (m')}{m'}\bigg)^{1/n} \\
& = \lim_{n \rightarrow \infty} (\ln\ln(e^n/b))^{1/n} = 1
\end{split}
\]
\end{proof}


\bibliography{report}
\bibliographystyle{ieeetr}

\end{document}
