\documentclass[12pt]{article}

\usepackage{amsfonts}
\usepackage{amsthm}
\usepackage{amsmath}
\usepackage{amssymb}

\newcommand{\floor}[1]{\lfloor #1 \rfloor}
\newcommand{\zz}{\mathbb{Z}}
\newcommand{\LL}{\mathcal{L}}

\newtheorem{question}{Question}
\newtheorem{lemma}{Lemma}
\newtheorem{remark}{Remark}
\newtheorem{definition}{Definition}


\title{Report(draft)}
\author{ Sasha }



\begin{document}

\maketitle

\section{Generalization of the construction for integers}
Plan:
\begin{enumerate}
    \item How to construct the lattice for non-prime modulus. Definition. (Maybe on why it's a lattice?)
    \item How does decoding algorithm work:
    \begin{itemize}
        \item parity check representation
        \item lattice as an intersection
        \item dual basis of the intersection
        \item gettin primal basis from dual
    \end{itemize}
    \item How to configure the parameters such that complexity is polynomial
    \item Argument about loglog n an that having more factors does not give us an advantage
\end{enumerate}

\subsection{Construction}
In this chapter we take $m = \prod_{i=1}^{k} q_{i}^{e_{i}}$ where $\{q_{i}\}$ are odd prime numbers and $\{e_{i}\}$ are positive integers. We work with a group $(\zz/m\zz)^*$ which is not cyclic anymore so we cannot deploy properties of discrete logarithm. Nevertheless, Chinese Remainder Theorem gives us the following equivalence:
\begin{equation}
    (\zz/m\zz)^* \sim \prod_{i=1}^{k}(\zz/q_{i}^{e_{i}}\zz)^*
\end{equation}
And for every prime $q_{i} > 2$ the group $(\zz/q_{i}^{e_{i}}\zz)^*$ is known to be cyclic. So now, we consider discrete logarithms in every component of the product to find a lattice basis.

Similarly to the initial construction we take a number $n$ and $B$ such that there exist $n$ prime numbers $p_{1}, ... , p_{n}$ different from every $q_{i}$ and bounded by $B$. And consider a group morphism:
\begin{equation}
    \psi : \zz^{n} \rightarrow (\zz/m\zz)^*
\end{equation}
\begin{equation}
    (x_{1}, ..., x_{n}) \mapsto \prod_{i=1}^{n}p_{i}^{x_{i}} \pmod{m}
\end{equation}

Lattice is defined as the kernel of the morphism:
\begin{equation}
    \LL = \ker \psi = \{(x_{1}, ..., x_{n}) \in \zz^{n} | \prod_{i=1}^{n}p_{i}^{x_{i}} \equiv 1 \pmod{m}\}
\end{equation}
Suppose we know for every $i$ a ${\beta_{j}}$ generator of  $(\zz/q_{j}^{e_{j}}\zz)^*$. Then appying CRT we get
\begin{equation}
    \LL = \ker \psi = \{(x_{1}, ..., x_{n}) \in \zz^{n} |  \forall 1 \leq j \leq k: \prod_{i=1}^{n}p_{i}^{x_{i}} \equiv 0 \pmod{q_{j}^{e_{j}}\}
\end{equation}


\end{document}
